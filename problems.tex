\documentclass[10pt]{article}
\usepackage[margin=1in]{geometry}
\title{Interesting Problems}
\author{Ryan Dougherty}
\date{}

\begin{document}

\maketitle

\begin{enumerate}
\section{Finite Automata}
\item Let $L$ be any unary language over $\Sigma = \{0\}$. Show that $L^*$ is regular.

\par\textbf{Solution:} consider $x = 0^a, y = 0^b$ with $\gcd(a, b) = 1$ (i.e., do not share any prime factors). With different combinations of uses of $a, b$, we can get any number larger than $(a-1)(b-1) - 1$. Therefore, $L^*$ is a union of a finite language (the combinations of $a, b$ that are less than $(a-1)(b-1)-1$), and $L_2 = \{w \colon |w| > (a-1)(b-1)-1\}$. $L_2$ is regular because it is just $\Sigma^*$ with a finite number of strings removed. Since $L^*$ is a union of regular languages, it is regular. 

\par If all $w \in L^*$ have lengths such that their $\gcd$ is $m$, we can just divide all string lengths by $m$ (and using the above argument repeatedly). $L^*$ now is the concatenation of some finite language and $L_3 = \{w \in (0^m)^* \colon |w| > m^2 \times [(\frac{x}{m}-1)(\frac{y}{m}-1) -1]\}$. Again, we are removing a finite number of words from $(0^m)^*$ to get $L_3$, so it is regular. 

\section{Context-Free Grammars}

\section{Turing Machines}

\end{enumerate}
\end{document}